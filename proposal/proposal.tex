\documentclass[12pt]{article}
\usepackage{fullpage}
\usepackage[left=1in,top=1in,right=1in,bottom=1in]{geometry}
\usepackage{titlesec}
\usepackage{enumerate}
\usepackage{amsmath}
\usepackage{amsthm}
\usepackage{paralist}
\usepackage{times}
\usepackage{cite}
\usepackage{hyperref} %for the URLs
\titleformat{\subsubsection}[runin]{}{}{}{}[]
\renewcommand{\labelenumi}{\alph{enumi}.}

\title{Semantic Analysis of Epilepsy-Related Patient Discharge Summaries}
\author{Ian Dimayuga, Tom Dooner \\icd3@case.edu, ted27@case.edu}
\begin{document}
\maketitle

\section{Background}
As defined by the NIH, Epilepsy is a brain disorder manifested in repeated seizures
over time.~\cite{nih-epilepsy} When patients have particularly bad seizures, they
are admitted to a hospital for monitoring. Upon admission to the hospital's Epilepsy
Monitoring Unit, various patient data is collected -- data such as Name, Sex, DOB,
and Age. The patient is also asked for some information regarding their recent seizure
episode such as zone, type, and origin.

This information, as well as the information collected from monitoring the patient
with EEGs during the hospital stay, is summarized in a discharge summary report.
These discharge reports are in a non-standard form and the data concerning background
information is in a natural-language form, making the data difficult to quickly
search.

We seek to codify and automatically determine the semantic relationships inherent
in patient discharge summaries to facilitate easy recall.

\subsection{Related Work}
The broader field of applying Natural Language Processing (NLP) to Electronic Health Records
is well-discussed in academic journals. Indeed, a Clinical Data Architecture has been
proposed~\cite{CDA} which promises to structure electronic health records. Unfortunately,
decades of health records exist digitally but in an unstructured format, so it is prudent
to consider using NLP technology to detect medical errors, correlate data, and draw
conclusions on vast quantities of data~\cite{friedman}.

Prior work exists in other NLP-related fields, such as evaluating how semantically similar 
two suggested medical terms are~\cite{Pedersen2007288} and how to analyze patient
discharge summaries to extract keywords and key relationships from the content of the
reports~\cite{soderland}. Perhaps the most relevant prior work is Uzuner et al., in 2006,
showed with some success that smoking status can be inferred from medical discharge 
records~\cite{Uzuner200814}.

While our goals are not as black-and-white as Uzuner et al., we hope to take advantage of
research contained in these referenced articles to produce a useful tool which will
help medical professionals draw conclusions by querying more-structured data.

\subsection{Significance}


\section{Approach}

\section{Timeline}

\section{References}
\bibliography{citations}{}
\bibliographystyle{plain}


\end{document}
